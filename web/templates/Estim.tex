\documentclass[french,11pt]{article}
\usepackage{babel}
\usepackage[T1]{fontenc}
\usepackage[utf8]{inputenc}
\usepackage[a4paper]{geometry}
\usepackage{units}
\usepackage{bera}
\usepackage{graphicx}
\usepackage{fancyhdr}
\usepackage{fp}

\def\TVA{20}    % Taux de la TVA

\def\TotalHT{0}
\def\TotalTVA{0}


\newcommand{\AjouterService}[2]{%    Arguments : Désignation, prix
    \FPround{\montant}{#2}{2}
    \FPadd{\TotalHT}{\TotalHT}{\montant}
    \eaddto\ListeProduits{#1    &   \montant    \cr}
}


\newcommand{\AfficheResultat}{%
    \ListeProduits
   
    \FPeval{\TotalTVA}{\TotalHT * \TVA / 100}
    \FPadd{\TotalTTC}{\TotalHT}{\TotalTVA}
    \FPround{\TotalHT}{\TotalHT}{2}
    \FPround{\TotalTVA}{\TotalTVA}{2}
    \FPround{\TotalTTC}{\TotalTTC}{2}
    \global\let\TotalHT\TotalHT
    \global\let\TotalTVA\TotalTVA
    \global\let\TotalTTC\TotalTTC
   
    \cr 
    \hline
    \textbf{Total}    & & &    \TotalHT
}

\newcommand*\eaddto[2]{% version développée de \addto
   \edef\tmp{#2}%
   \expandafter\addto
   \expandafter#1%
   \expandafter{\tmp}%
}

\newcommand{\ListeProduits}{}




%%%%%%%%%%%%%%%%%%%%% A MODIFIER DANS LA FACTURE %%%%%%%%%%%%%%%%%%%%%

\def\FactureNum            {<none>}    % Numéro de facture
\def\FactureAcquittee    {non}        % Facture acquittée : oui/non
\def\FactureLieu    {Suresnes}    % Lieu de l'édition de la facture
\def\FactureObjet    {Facture}    % Objet du document
% Description de la facture
\def\FactureDescr    {%

Cette facture concerne la prestation de coiffure a domicile du Mardi 23 Septembre 2014 à Meudon, 6 rue de la Verrerie.
}

% Infos Client
\def\ClientNom{M. Dupont}    % Nom du client
\def\ClientAdresse{%                    % Adresse du client
   Zenosphere : Anthony Courtois (DG)\\
   6 rue de la Verrerie\\
   92190 MEUDON\\
   Mobile: 06 47 60 25 50     
}

% Liste des produits facturés : Désignation, prix
\AjouterService    {Forfait Coiffure}        {75}
%\AjouterService    {Frais de déplacement} {0.84}
%%%%%%%%%%%%%%%%%%%%%%%%%%%%%%%%%%%%%%%%%%%%%%%%%%%%%%%%%%%%%%%%%%%%%%




\geometry{verbose,tmargin=4em,bmargin=8em,lmargin=6em,rmargin=6em}
\setlength{\parindent}{0pt}
\setlength{\parskip}{1ex plus 0.5ex minus 0.2ex}

\thispagestyle{fancy}
\pagestyle{fancy}
\setlength{\parindent}{0pt}

\renewcommand{\headrulewidth}{0pt}
\cfoot{
    Soraya Coiffure - 2 boulevard Aristide Briand - 92150 SURESNES \newline
    \small{
    E-mail: soraya.boudjouraf@free.fr\\
    Téléphone mobile: +33(0)6 37 57 42 74\\
        Téléphone fixe: +33(0)9 80 90 36 42
    }
}



\begin{document}

% Logo de la société
%\includegraphics{logo.jpg}

% Nom et adresse de la société
Soraya Schneider\\
2 boulevard Aristide Briand\\
Bat V\\
92150 SURESNES\\

Facture n°\FactureNum


{\addtolength{\leftskip}{10.5cm} %in ERT
    \textbf{\ClientNom}    \\
    \ClientAdresse        \\

} %in ERT


\hspace*{10.5cm}
\FactureLieu, le \today

~\\~\\

\textbf{Objet : \FactureObjet \\}

\textnormal{\FactureDescr}

~\\

\begin{center}
    \begin{tabular}{lrrr}
        \textbf{Désignation ~~~~~~}   & \textbf{Montant (EUR)}    \\
        \hline
        \AfficheResultat{}
    \end{tabular}
\end{center}

\begin{flushright}
\textit{Auto entreprise en franchise de TVA}\\

\end{flushright}
~\\

\ifthenelse{\equal{\FactureAcquittee}{oui}}{
    Facture acquittée.
}{

    À régler par chèque ou par virement bancaire :

    \begin{center}
        \begin{tabular}{|c c c c|}
            \hline     \textbf{Code banque}    & \textbf{Code guichet}    & \textbf{N° de Compte}        & \textbf{Clé RIB}    \\
                    20041                    & 00001                     & 1225647F020                & 05               \\
            \hline     \textbf{IBAN N°}        & \multicolumn{3}{|l|}{ FR 91 20041 00001 1225647F020 05 }         \\
            \hline     \textbf{Code BIC}        & \multicolumn{3}{|l|}{ PSSTFRPPPAR }         \\
            \hline
        \end{tabular}
    \end{center}

}

\end{document}
